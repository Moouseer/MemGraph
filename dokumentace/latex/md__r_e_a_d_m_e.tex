Mem\+Graph je knihovna pro tvorbu grafické reprezentace stavů paměti programu. Je určena především pro verifikační nástroj \href{http://www.fit.vutbr.cz/research/groups/verifit/tools/predator/}{\tt Predator}, případně \href{http://www.fit.vutbr.cz/research/groups/verifit/tools/forester/}{\tt Forester}. Tyto dva nástroje jsou vyvíjeny výzkumnou skupinou \href{http://www.fit.vutbr.cz/research/groups/verifit/.cs}{\tt veri\+F\+IT} na fakultě informačních technologií V\+UT v Brně.

Knihovna implementuje tvorbu grafových struktur a jejich následnou vizualizaci objektově orientovaným přístupem v jazyce C++(11) a je k dispozici k dalším úpravám a libovolnému použití pod licencí G\+NU G\+P\+Lv3.

\subsection*{Vlastnosti knihovny}

Knihovna je rozdělena na dvě části -\/ datový model a funkční část.

Datový model poskytuje sadu metod pro tvorbu grafových struktur. Je inspirovaný prvky grafu, které popisuje teorie grafu -\/ hrany, vrcholy a podgrafy.

Funkční část zpracovává datový model a transformuje jej do požadované podoby. V případě knohovny je možné provést transormace do struktury jazyka D\+OT, případně vytvořit vizuální reprezentaci struktury. Pro tvorbu vizuální reprezentace je použita knihovna \href{http://www.graphviz.org}{\tt Graphviz}. V datovém modelu je možné jednotlivým prvkům grafu přidávat atributy, které představují vizuální informaci pro vykreslení nástrojem Graphviz.

\subsubsection*{Dostupné funkce knihovny}


\begin{DoxyItemize}
\item Výpis datového modelu ve struktuře D\+OT
\item Grafický výstup datového modelu
\item Načtení struktury D\+OT a převod do datového modelu knihovny
\item Transformace datového modelu do jiné podoby (level-\/of-\/details některých prvků nástroje Predator -\/ pouze na ukázku)
\end{DoxyItemize}

\subsubsection*{V plánu}


\begin{DoxyItemize}
\item Napojení na verifikační nástroj Forester
\item Porovnání dvou grafových struktur (izomorfismus)
\item Vizuální styly verifikačního nástroje Predator
\end{DoxyItemize}

\subsection*{Příklad použití}

Příklad tvorby datového modelu \begin{DoxyVerb}// vytvori novy graf
Graph *graph = new Graph();

// nastavi grafu atribut
graph->setAttr("nodesep",1);

Attributes node_attrs;
Attributes edge_attrs;

// nastaveni defaultni atributu vrcholu grafu
node_attrs.setAttr("color","red");
node_attrs.setAttr("fontname","Courier");
node_attrs.setAttr("shape","box");

// nastaveni defaultni atributu hran grafu
edge_attrs.setAttr("color","blue");
edge_attrs.setAttr("style","dashed");

// predani atributu instanci grafu
graph->setNodeAttrs(&node_attrs);
graph->setEdgeAttrs(&edge_attrs);

// pridani hran
graph->addEdge("vutbr", "fit");
graph->addEdge("vutbr", "fce");
graph->addEdge("vutbr", "feec");
graph->addEdge("fit", "merlin");
graph->addEdge("fit", "kazi");
graph->addEdge("feec", "kos");
\end{DoxyVerb}


Generování struktury D\+OT z definovaného datového modelu \begin{DoxyVerb}// DOT generujeme volanim metody getDot() na instanci tridy GraphvizPlotter

// vytvoreni instance tridy GraphvizPlotter, kteremu predame definovany datovy model    
GraphvizPlotter *plotter = new GraphvizPlotter(graph);

// volani metody getDot()
plotter->getDot();
\end{DoxyVerb}


Dostaneme následující D\+OT strukturu \begin{DoxyVerb}digraph G {
    nodesep=1;

    node [color=red,fontname=Courier,shape=box];

    edge [color=blue,style=dashed];

    vutbr -> fit;
    vutbr -> fce;
    vutbr -> feec;
    fit -> merlin;
    fit -> kazi;
    feec -> kos;
}
\end{DoxyVerb}


Pro generování vizuální reprezentace nastavíme pomocí metody set\+Output\+Path(std\+::string path) cestu uložení souboru, metodou set\+Output\+Name(std\+::string name) jméno souboru a set\+Output\+Format(\+Graphviz\+Plotter\+::output format) výsledný formát souborů, ve kterém bude graf uložen. K dispozici jsou standardní formáty jako png, pdf, svg apod. Graf vykreslíme voláním metody plot(). Všechny tyto metody jsou veřejnými metodamy třídy Graphviz\+Plotter \begin{DoxyVerb}// ./example.png
plotter->setOutputPath("./");
plotter->setOutputFormat(GraphvizPlotter::PNG);
plotter->setOutputName("example");
plotter->plot();
\end{DoxyVerb}


Výsledek grafického výstupu reprezentuje níže uvedený obrázek



Je vhodné dodat, že tuto knihovnu není nutné použít pouze v návaznosti na verifikační nástroje Predator a Forester. Díky univerzálnosti datového modelu je možné definovat jakékoli grafové struktury.

\subsection*{Závislosti knihovny}

Knihovna je určena především pro distribuce systému Linux, respektive Ubuntu. Nicméně je otestována a plně funkční také pod O\+SX El Capitan. Pod os Windows 7 byla knihovna pro účely testování pouze přeložena a sestavena. Pro správnou funkčnost knihovny je nutné mít v systému Linux dostupné tyto nástroje a knihovny (v této konfiguraci otestováno)\+:


\begin{DoxyItemize}
\item G\+CC 4.\+9.\+3
\item C\+Make 2.\+8
\item libgraphviz-\/dev
\end{DoxyItemize}

\subsection*{Nasazení na verifikační nástroj Predator}

Vnořte se do složky, do které chcete stáhnout verifikační nástroj Predator a knihovnu Mem\+Graph. Následně stáhněte nástroj Predator \begin{DoxyVerb}git clone https://github.com/kdudka/predator.git
\end{DoxyVerb}


a knihovnu Mem\+Graph \begin{DoxyVerb}git clone https://github.com/Moouseer/MemGraph.git
\end{DoxyVerb}


Nástroj Predator pro svoji korektní funkci potřebuje závisloti překladače G\+CC. Níže jsou uvedeny příkazy pro doinstalování závislostí pro G\+CC 4.\+9 \begin{DoxyVerb}sudo apt-get install g++-4.9-multilib
sudo apt-get install gcc-4.9-plugin-dev
sudo apt-get install libboost-all-dev
sudo apt-get install cmake
\end{DoxyVerb}


Nyní nainstalujeme závistlosti knihovny Mem\+Graph \begin{DoxyVerb}sudo apt-get install libgraphviz-dev
\end{DoxyVerb}


Nahradíme některé nezbytné soubory z repozitáře nástroje Predator soubory knihovny (C\+Make\+Files, symplot.\+cc, ...). \begin{DoxyVerb}cd MemGraph/

find . -name "*.cpp" -type f -not -path "./predator/sl/Memgraph/*" -exec cp {} predator/sl/Memgraph/ \;

find . -name "*.h" -type f -not -path "./predator/sl/Memgraph/*" -exec cp {} predator/sl/Memgraph/ \;

cp -R ./predator/sl ../predator/
\end{DoxyVerb}


Nástroj Predator přeložíme a sestavíme společně s knihovnou Mem\+Graph \begin{DoxyVerb}cd ../predator/
./switch-host-gcc.sh /usr/bin/gcc-4.9
\end{DoxyVerb}


Po překladu a sestavení frameworku Code Listener je spuštěn překlad a sestavení knihovny Mem\+Graph a následně překlad a sestavení verifikačního nástroje Predator. Po překladu a sestavení nástroje Predator je spuštěna sada 849 testů. Je nutné říci, že při sestavení s knihovnou Mem\+Graph neprojdou celkem tři testy. Tyto testy nejsou dokončeny a tak se testovací skript zasekne u 846 testu a čeká na výsledky. Toto čekání je nutné přerušit pomocí zkratky ctrl + c.

Testy ve složce sl\+\_\+build tvoří textové a grafické výstupy verifikace. Výstupy jsou tvořeny následnovně\+:

Predator


\begin{DoxyItemize}
\item Generuje soubory s příponou dot, které v názvu souboru neobsahují řetězec bc\+Plot
\end{DoxyItemize}

Mem\+Graph


\begin{DoxyItemize}
\item Generuje soubory s příponou dot, které v názvu souboru obsahují řetězec bc\+Plot (výsledek metody get\+Dot())
\item Generuje soubory s příponout png, které v názvu souboru obsahují řetězec bc\+Plot (výsledek metody plot())
\item Generuje soubory s příponout png, které v názvu souboru obsahují řetězec parsed\+\_\+bc\+Plot (načtení původního dot souboru z nástroje Predator, vložení obsahu do metody parse\+Dot() a následné vytvoření grafu pomoci metody plot())
\item Generuje soubory s příponout png, které v názvu souboru obsahují řetězec zabstract\+\_\+bc\+Plot (provedení vzorových transformací nad objekty S\+LS a D\+LS a vrcholu Value S\+MG grafu) 
\end{DoxyItemize}